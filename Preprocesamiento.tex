\PassOptionsToPackage{unicode=true}{hyperref} % options for packages loaded elsewhere
\PassOptionsToPackage{hyphens}{url}
%
\documentclass[]{article}
\usepackage{lmodern}
\usepackage{amssymb,amsmath}
\usepackage{ifxetex,ifluatex}
\usepackage{fixltx2e} % provides \textsubscript
\ifnum 0\ifxetex 1\fi\ifluatex 1\fi=0 % if pdftex
  \usepackage[T1]{fontenc}
  \usepackage[utf8]{inputenc}
  \usepackage{textcomp} % provides euro and other symbols
\else % if luatex or xelatex
  \usepackage{unicode-math}
  \defaultfontfeatures{Ligatures=TeX,Scale=MatchLowercase}
\fi
% use upquote if available, for straight quotes in verbatim environments
\IfFileExists{upquote.sty}{\usepackage{upquote}}{}
% use microtype if available
\IfFileExists{microtype.sty}{%
\usepackage[]{microtype}
\UseMicrotypeSet[protrusion]{basicmath} % disable protrusion for tt fonts
}{}
\IfFileExists{parskip.sty}{%
\usepackage{parskip}
}{% else
\setlength{\parindent}{0pt}
\setlength{\parskip}{6pt plus 2pt minus 1pt}
}
\usepackage{hyperref}
\hypersetup{
            pdftitle={Preprocesamiento de Datos Modelo de Clasificación Prevención de Fraude},
            pdfauthor={Ing. Juan Sebastián Arboleda},
            pdfborder={0 0 0},
            breaklinks=true}
\urlstyle{same}  % don't use monospace font for urls
\usepackage[margin=1in]{geometry}
\usepackage{color}
\usepackage{fancyvrb}
\newcommand{\VerbBar}{|}
\newcommand{\VERB}{\Verb[commandchars=\\\{\}]}
\DefineVerbatimEnvironment{Highlighting}{Verbatim}{commandchars=\\\{\}}
% Add ',fontsize=\small' for more characters per line
\usepackage{framed}
\definecolor{shadecolor}{RGB}{248,248,248}
\newenvironment{Shaded}{\begin{snugshade}}{\end{snugshade}}
\newcommand{\AlertTok}[1]{\textcolor[rgb]{0.94,0.16,0.16}{#1}}
\newcommand{\AnnotationTok}[1]{\textcolor[rgb]{0.56,0.35,0.01}{\textbf{\textit{#1}}}}
\newcommand{\AttributeTok}[1]{\textcolor[rgb]{0.77,0.63,0.00}{#1}}
\newcommand{\BaseNTok}[1]{\textcolor[rgb]{0.00,0.00,0.81}{#1}}
\newcommand{\BuiltInTok}[1]{#1}
\newcommand{\CharTok}[1]{\textcolor[rgb]{0.31,0.60,0.02}{#1}}
\newcommand{\CommentTok}[1]{\textcolor[rgb]{0.56,0.35,0.01}{\textit{#1}}}
\newcommand{\CommentVarTok}[1]{\textcolor[rgb]{0.56,0.35,0.01}{\textbf{\textit{#1}}}}
\newcommand{\ConstantTok}[1]{\textcolor[rgb]{0.00,0.00,0.00}{#1}}
\newcommand{\ControlFlowTok}[1]{\textcolor[rgb]{0.13,0.29,0.53}{\textbf{#1}}}
\newcommand{\DataTypeTok}[1]{\textcolor[rgb]{0.13,0.29,0.53}{#1}}
\newcommand{\DecValTok}[1]{\textcolor[rgb]{0.00,0.00,0.81}{#1}}
\newcommand{\DocumentationTok}[1]{\textcolor[rgb]{0.56,0.35,0.01}{\textbf{\textit{#1}}}}
\newcommand{\ErrorTok}[1]{\textcolor[rgb]{0.64,0.00,0.00}{\textbf{#1}}}
\newcommand{\ExtensionTok}[1]{#1}
\newcommand{\FloatTok}[1]{\textcolor[rgb]{0.00,0.00,0.81}{#1}}
\newcommand{\FunctionTok}[1]{\textcolor[rgb]{0.00,0.00,0.00}{#1}}
\newcommand{\ImportTok}[1]{#1}
\newcommand{\InformationTok}[1]{\textcolor[rgb]{0.56,0.35,0.01}{\textbf{\textit{#1}}}}
\newcommand{\KeywordTok}[1]{\textcolor[rgb]{0.13,0.29,0.53}{\textbf{#1}}}
\newcommand{\NormalTok}[1]{#1}
\newcommand{\OperatorTok}[1]{\textcolor[rgb]{0.81,0.36,0.00}{\textbf{#1}}}
\newcommand{\OtherTok}[1]{\textcolor[rgb]{0.56,0.35,0.01}{#1}}
\newcommand{\PreprocessorTok}[1]{\textcolor[rgb]{0.56,0.35,0.01}{\textit{#1}}}
\newcommand{\RegionMarkerTok}[1]{#1}
\newcommand{\SpecialCharTok}[1]{\textcolor[rgb]{0.00,0.00,0.00}{#1}}
\newcommand{\SpecialStringTok}[1]{\textcolor[rgb]{0.31,0.60,0.02}{#1}}
\newcommand{\StringTok}[1]{\textcolor[rgb]{0.31,0.60,0.02}{#1}}
\newcommand{\VariableTok}[1]{\textcolor[rgb]{0.00,0.00,0.00}{#1}}
\newcommand{\VerbatimStringTok}[1]{\textcolor[rgb]{0.31,0.60,0.02}{#1}}
\newcommand{\WarningTok}[1]{\textcolor[rgb]{0.56,0.35,0.01}{\textbf{\textit{#1}}}}
\usepackage{graphicx,grffile}
\makeatletter
\def\maxwidth{\ifdim\Gin@nat@width>\linewidth\linewidth\else\Gin@nat@width\fi}
\def\maxheight{\ifdim\Gin@nat@height>\textheight\textheight\else\Gin@nat@height\fi}
\makeatother
% Scale images if necessary, so that they will not overflow the page
% margins by default, and it is still possible to overwrite the defaults
% using explicit options in \includegraphics[width, height, ...]{}
\setkeys{Gin}{width=\maxwidth,height=\maxheight,keepaspectratio}
\setlength{\emergencystretch}{3em}  % prevent overfull lines
\providecommand{\tightlist}{%
  \setlength{\itemsep}{0pt}\setlength{\parskip}{0pt}}
\setcounter{secnumdepth}{0}
% Redefines (sub)paragraphs to behave more like sections
\ifx\paragraph\undefined\else
\let\oldparagraph\paragraph
\renewcommand{\paragraph}[1]{\oldparagraph{#1}\mbox{}}
\fi
\ifx\subparagraph\undefined\else
\let\oldsubparagraph\subparagraph
\renewcommand{\subparagraph}[1]{\oldsubparagraph{#1}\mbox{}}
\fi

% set default figure placement to htbp
\makeatletter
\def\fps@figure{htbp}
\makeatother


\title{Preprocesamiento de Datos Modelo de Clasificación Prevención de Fraude}
\author{Ing. Juan Sebastián Arboleda}
\date{21 de octubre de 2021}

\begin{document}
\maketitle

\hypertarget{lectura-de-datos}{%
\section{Lectura de datos}\label{lectura-de-datos}}

\begin{Shaded}
\begin{Highlighting}[]
\KeywordTok{library}\NormalTok{(readr)}
\NormalTok{Fraud_Dataset_Data_ <-}\StringTok{ }\KeywordTok{read_csv}\NormalTok{(}\StringTok{"Fraud Dataset  - Data .csv"}\NormalTok{, }
    \DataTypeTok{col_types =} \KeywordTok{cols}\NormalTok{(}\DataTypeTok{Fraude =} \KeywordTok{col_integer}\NormalTok{()), }
    \DataTypeTok{locale =} \KeywordTok{locale}\NormalTok{())}

\NormalTok{Data <-}\StringTok{ }\NormalTok{Fraud_Dataset_Data_}
\end{Highlighting}
\end{Shaded}

\hypertarget{exploraciuxf3n-de-los-datos}{%
\section{Exploración de los datos}\label{exploraciuxf3n-de-los-datos}}

\begin{Shaded}
\begin{Highlighting}[]
\CommentTok{# Dimensión}
\KeywordTok{dim}\NormalTok{(Data)}
\end{Highlighting}
\end{Shaded}

\begin{verbatim}
## [1] 16880    21
\end{verbatim}

\begin{Shaded}
\begin{Highlighting}[]
\CommentTok{# Resumen de los datos}
\KeywordTok{summary}\NormalTok{(Data)}
\end{Highlighting}
\end{Shaded}

\begin{verbatim}
##        A                 B                C                D           
##  Min.   : 0.0000   Min.   :-1.000   Min.   :     0   Min.   :  0.0000  
##  1st Qu.: 0.0000   1st Qu.: 4.000   1st Qu.:  1172   1st Qu.:  0.0000  
##  Median : 0.0000   Median : 7.000   Median :  6173   Median :  0.0000  
##  Mean   : 0.3092   Mean   : 7.645   Mean   : 39235   Mean   :  0.1987  
##  3rd Qu.: 0.0000   3rd Qu.:11.000   3rd Qu.: 26889   3rd Qu.:  0.0000  
##  Max.   :30.0000   Max.   :20.000   Max.   :617324   Max.   :180.0000  
##                                     NA's   :3197                       
##        E                 F                 G                H          
##  Min.   : 0.0000   Min.   :0.00000   Min.   :0.0000   Min.   : 0.0000  
##  1st Qu.: 0.0000   1st Qu.:0.00000   1st Qu.:0.0000   1st Qu.: 0.0000  
##  Median : 0.0000   Median :0.00000   Median :0.0000   Median : 0.0000  
##  Mean   : 0.4337   Mean   :0.01588   Mean   :0.0052   Mean   : 0.0503  
##  3rd Qu.: 0.0000   3rd Qu.:0.00000   3rd Qu.:0.0000   3rd Qu.: 0.0000  
##  Max.   :45.0000   Max.   :1.00000   Max.   :1.0000   Max.   :21.0000  
##                                                                        
##        I                J                   K               L         
##  Min.   : 0.0000   Length:16880       Min.   :0.120   Min.   :0.0000  
##  1st Qu.: 0.0000   Class :character   1st Qu.:0.580   1st Qu.:0.0000  
##  Median : 0.0000   Mode  :character   Median :0.680   Median :0.0000  
##  Mean   : 0.1441                      Mean   :0.682   Mean   :0.4323  
##  3rd Qu.: 0.0000                      3rd Qu.:0.800   3rd Qu.:1.0000  
##  Max.   :24.0000                      Max.   :0.990   Max.   :7.0000  
##                                       NA's   :12864                   
##        M                N                O                  P         
##  Min.   : 1.000   Min.   : 1.000   Min.   :0.000000   Min.   : 1.000  
##  1st Qu.: 1.000   1st Qu.: 1.000   1st Qu.:0.000000   1st Qu.: 1.000  
##  Median : 1.000   Median : 1.000   Median :0.000000   Median : 1.000  
##  Mean   : 1.544   Mean   : 1.092   Mean   :0.009419   Mean   : 1.631  
##  3rd Qu.: 2.000   3rd Qu.: 1.000   3rd Qu.:0.000000   3rd Qu.: 2.000  
##  Max.   :13.000   Max.   :10.000   Max.   :3.000000   Max.   :41.000  
##                                                                       
##        Q                  R                  S             Monto         
##  Min.   :   0.000   Min.   :   0.000   Min.   :-1.00   Min.   :    0.05  
##  1st Qu.:   0.000   1st Qu.:   0.000   1st Qu.: 9.56   1st Qu.:   33.81  
##  Median :   0.000   Median :   0.000   Median :20.64   Median :   81.64  
##  Mean   :   8.445   Mean   :   1.995   Mean   :29.13   Mean   :  161.84  
##  3rd Qu.:   0.000   3rd Qu.:   0.000   3rd Qu.:39.21   3rd Qu.:  193.44  
##  Max.   :2274.670   Max.   :2025.720   Max.   :99.97   Max.   :12538.44  
##                                                                          
##      Fraude      
##  Min.   :0.0000  
##  1st Qu.:0.0000  
##  Median :0.0000  
##  Mean   :0.2732  
##  3rd Qu.:1.0000  
##  Max.   :1.0000  
## 
\end{verbatim}

\begin{Shaded}
\begin{Highlighting}[]
\KeywordTok{library}\NormalTok{(corrplot)}
\end{Highlighting}
\end{Shaded}

\begin{verbatim}
## corrplot 0.90 loaded
\end{verbatim}

\begin{Shaded}
\begin{Highlighting}[]
\KeywordTok{library}\NormalTok{(dplyr)}
\end{Highlighting}
\end{Shaded}

\begin{verbatim}
## 
## Attaching package: 'dplyr'
\end{verbatim}

\begin{verbatim}
## The following objects are masked from 'package:stats':
## 
##     filter, lag
\end{verbatim}

\begin{verbatim}
## The following objects are masked from 'package:base':
## 
##     intersect, setdiff, setequal, union
\end{verbatim}

\begin{Shaded}
\begin{Highlighting}[]
\CommentTok{#Un valor menor que 0 indica que existe una correlación negativa}
\CommentTok{#Un valor mayor que 0 indica que existe una correlación positiva}
\CommentTok{#una correlación de 0, o próxima a 0, indica que no hay relación lineal entre las dos variables}

\NormalTok{Data_num <-}\StringTok{ }\KeywordTok{select}\NormalTok{(Data,}\OperatorTok{-}\NormalTok{J, }\OperatorTok{-}\NormalTok{Fraude)}
\KeywordTok{corrplot}\NormalTok{(}\KeywordTok{cor}\NormalTok{(Data_num),        }\CommentTok{# Matriz de correlación}
         \DataTypeTok{method =} \StringTok{"shade"}\NormalTok{, }\CommentTok{# Método para el gráfico de correlación}
         \DataTypeTok{type =} \StringTok{"full"}\NormalTok{,    }\CommentTok{# Estilo del gráfico (también "upper" y "lower")}
         \DataTypeTok{diag =} \OtherTok{TRUE}\NormalTok{,      }\CommentTok{# Si TRUE (por defecto), añade la diagonal}
         \DataTypeTok{tl.col =} \StringTok{"black"}\NormalTok{, }\CommentTok{# Color de las etiquetas}
         \DataTypeTok{bg =} \StringTok{"white"}\NormalTok{,     }\CommentTok{# Color de fondo}
         \DataTypeTok{title =} \StringTok{""}\NormalTok{,       }\CommentTok{# Título}
         \DataTypeTok{col =} \OtherTok{NULL}\NormalTok{)       }\CommentTok{# Paleta de colores}
\end{Highlighting}
\end{Shaded}

\includegraphics{Preprocesamiento_files/figure-latex/unnamed-chunk-4-1.pdf}

\begin{Shaded}
\begin{Highlighting}[]
\CommentTok{#correlacion}
\KeywordTok{library}\NormalTok{(psych)}

\KeywordTok{corPlot}\NormalTok{(Data_num[ ,}\DecValTok{4}\OperatorTok{:}\DecValTok{9}\NormalTok{], }\DataTypeTok{cex =} \FloatTok{1.2}\NormalTok{, }\DataTypeTok{main =} \StringTok{"Matriz de correlación")}
\end{Highlighting}
\end{Shaded}

\includegraphics{Preprocesamiento_files/figure-latex/unnamed-chunk-5-1.pdf}

\begin{Shaded}
\begin{Highlighting}[]
\KeywordTok{corPlot}\NormalTok{(Data_num[ ,}\DecValTok{11}\OperatorTok{:}\DecValTok{15}\NormalTok{], }\DataTypeTok{cex =} \FloatTok{1.2}\NormalTok{, }\DataTypeTok{main =} \StringTok{"Matriz de correlación")}
\end{Highlighting}
\end{Shaded}

\includegraphics{Preprocesamiento_files/figure-latex/unnamed-chunk-5-2.pdf}

\begin{Shaded}
\begin{Highlighting}[]
\CommentTok{# Cantidad de registros con información del tipo NA}
\KeywordTok{sum}\NormalTok{(}\KeywordTok{is.na}\NormalTok{(Data))}
\end{Highlighting}
\end{Shaded}

\begin{verbatim}
## [1] 16061
\end{verbatim}

\begin{Shaded}
\begin{Highlighting}[]
\CommentTok{# Identificamos en que columna estan los datos faltantes}
\KeywordTok{apply}\NormalTok{(}\DataTypeTok{X =} \KeywordTok{is.na}\NormalTok{(Data), }\DataTypeTok{MARGIN =} \DecValTok{2}\NormalTok{, }\DataTypeTok{FUN =}\NormalTok{ sum)}
\end{Highlighting}
\end{Shaded}

\begin{verbatim}
##      A      B      C      D      E      F      G      H      I      J      K 
##      0      0   3197      0      0      0      0      0      0      0  12864 
##      L      M      N      O      P      Q      R      S  Monto Fraude 
##      0      0      0      0      0      0      0      0      0      0
\end{verbatim}

\begin{Shaded}
\begin{Highlighting}[]
\CommentTok{# Identificamos en que columna estan los datos faltantes}
\KeywordTok{summary}\NormalTok{(Data}\OperatorTok{$}\NormalTok{C)}
\end{Highlighting}
\end{Shaded}

\begin{verbatim}
##    Min. 1st Qu.  Median    Mean 3rd Qu.    Max.    NA's 
##       0    1172    6173   39235   26889  617324    3197
\end{verbatim}

\begin{Shaded}
\begin{Highlighting}[]
\KeywordTok{summary}\NormalTok{(Data}\OperatorTok{$}\NormalTok{K)}
\end{Highlighting}
\end{Shaded}

\begin{verbatim}
##    Min. 1st Qu.  Median    Mean 3rd Qu.    Max.    NA's 
##   0.120   0.580   0.680   0.682   0.800   0.990   12864
\end{verbatim}

\hypertarget{preprocesamiento-de-los-datos}{%
\section{Preprocesamiento de los
datos}\label{preprocesamiento-de-los-datos}}

Dado que en la fase de exploración identifiqué la presencia de registros
del tipo NA en las variables C y K, procedo a realizar algunas
transformaciones buscando mejorar la calidad de los mismos. Para el caso
de la Variable C, en donde la presencia de faltantes no supera el 20\%
de los datos, vamos a completar esos faltantes con el valor del segundo
quartil (la mediana). El valor promedio tambien es utilizado para ello,
pero en este caso no es el mas acertado dado que la diferencia entre el
tercer y cuarto quartil es muy amplia, lo que indica valores muy
extremos que pueden afectar el valor promedio.

\begin{Shaded}
\begin{Highlighting}[]
\NormalTok{Data}\OperatorTok{$}\NormalTok{C <-}\StringTok{ }\KeywordTok{ifelse}\NormalTok{(}\KeywordTok{is.na}\NormalTok{(Data}\OperatorTok{$}\NormalTok{C), }\KeywordTok{median}\NormalTok{(Data}\OperatorTok{$}\NormalTok{C, }\DataTypeTok{na.rm =} \OtherTok{TRUE}\NormalTok{), Data}\OperatorTok{$}\NormalTok{C)}
\KeywordTok{summary}\NormalTok{(Data}\OperatorTok{$}\NormalTok{C)}
\end{Highlighting}
\end{Shaded}

\begin{verbatim}
##    Min. 1st Qu.  Median    Mean 3rd Qu.    Max. 
##       0    1920    6173   32974   18570  617324
\end{verbatim}

Nota: Observemos que luego del reemplazo la media de los datos no se
afectó y la distribución de la variable mejoró.

Para el caso de la variable K, en donde la cantidad de datos faltantes
supera el 75\% de sus datos, opté por no tener en cuenta esa variable en
el análisis, dado que no aportaria confiabilidad.

\begin{Shaded}
\begin{Highlighting}[]
\KeywordTok{require}\NormalTok{(dplyr)}
\NormalTok{Data <-}\StringTok{ }\KeywordTok{select}\NormalTok{(Data, }\OperatorTok{-}\NormalTok{K)}
\NormalTok{Data}
\end{Highlighting}
\end{Shaded}

\begin{verbatim}
## # A tibble: 16,880 x 20
##        A     B     C     D     E     F     G     H     I J         L     M     N
##    <dbl> <dbl> <dbl> <dbl> <dbl> <dbl> <dbl> <dbl> <dbl> <chr> <dbl> <dbl> <dbl>
##  1     0    10 50257     0     0     0     0     0     0 UY        0     3     1
##  2     0    10 29014     0     0     0     0     0     0 UY        0     1     1
##  3     0     7    92     0     1     0     0     0     1 UY        0     3     1
##  4     9    16 50269     0     0     0     0     0     0 UY        0     3     1
##  5     0     8  8180     0     0     0     0     0     0 UY        0     1     1
##  6     1    12  1141     0     0     0     0     0     0 UY        0     1     1
##  7    10    18  4560     0     0     0     0     0     0 UY        0     3     1
##  8     0    10    38     0     0     0     0     0     0 UY        0     1     1
##  9     0    14  3790     0     0     0     0     0     0 UY        0     1     1
## 10     0    16 23210     0     0     0     0     0     0 UY        1     1     1
## # ... with 16,870 more rows, and 7 more variables: O <dbl>, P <dbl>, Q <dbl>,
## #   R <dbl>, S <dbl>, Monto <dbl>, Fraude <int>
\end{verbatim}

Continuando con el analisis, veo que la columna J es la única del tipo
categórico y asumo hace referencia al país.

\begin{Shaded}
\begin{Highlighting}[]
\KeywordTok{table}\NormalTok{(Data}\OperatorTok{$}\NormalTok{J)  }
\end{Highlighting}
\end{Shaded}

\begin{verbatim}
## 
##   AR   AU   BR   CA   CH   CL   CO   ES   FR   GB   GT   IT   KR   MX   PT   TR 
## 9329    1 4428   12    1    1    1  314    2    8    2    1    1 2366    1    1 
##   UA   US   UY 
##    1  230  180
\end{verbatim}

La gran mayoria de los datos estan agrupados en solo 3 países, el resto
suman una particpacion menor al 5\%, por ello veo conveniente agruparlos
en una sola categoria.

\begin{Shaded}
\begin{Highlighting}[]
\CommentTok{#Data$J [Data$J== c ("AU","CA","CH")] <- "Otros"}
\CommentTok{#Data$J <- ifelse(Data$J == c("AU","CA","CH","CL","CO","ES","FR","GB","GT","IT","KR","PT","TR","UA","US","UY"), "Otros",Data$J) }
\NormalTok{Otros =}\StringTok{ }\KeywordTok{c}\NormalTok{ (}\StringTok{"AR"}\NormalTok{,}\StringTok{"BR"}\NormalTok{,}\StringTok{"MX"}\NormalTok{)}
\NormalTok{Data}\OperatorTok{$}\NormalTok{J <-}\StringTok{ }\KeywordTok{ifelse}\NormalTok{(Data}\OperatorTok{$}\NormalTok{J }\OperatorTok\StringTok{ }\NormalTok{Otros, Data}\OperatorTok{$}\NormalTok{J,}\StringTok{"Otros"}\NormalTok{)}
\KeywordTok{table}\NormalTok{(Data}\OperatorTok{$}\NormalTok{J)  }
\end{Highlighting}
\end{Shaded}

\begin{verbatim}
## 
##    AR    BR    MX Otros 
##  9329  4428  2366   757
\end{verbatim}

Buscando un formato mas amigable para los algoritmos, implemento la
codificación one-hot encoding, de esta manera todas las variables
quedaran bajo un formato numérico. Por último elimino la columna J
original y escribo el archivo de salida con la data procesada que será
modelada en python.

\begin{Shaded}
\begin{Highlighting}[]
\KeywordTok{library}\NormalTok{(fastDummies)}
\NormalTok{Data <-}\StringTok{ }\KeywordTok{dummy_cols}\NormalTok{(Data, }\DataTypeTok{select_columns =} \KeywordTok{c}\NormalTok{(}\StringTok{"J"}\NormalTok{))}
\NormalTok{Data <-}\StringTok{ }\KeywordTok{select}\NormalTok{(Data, }\OperatorTok{-}\NormalTok{J)}
\end{Highlighting}
\end{Shaded}

\begin{Shaded}
\begin{Highlighting}[]
\KeywordTok{write.csv}\NormalTok{(Data, }\StringTok{"Data_Limpia.csv"}\NormalTok{,}\DataTypeTok{sep =} \StringTok{","}\NormalTok{,}\DataTypeTok{dec =} \StringTok{"."}\NormalTok{,}\DataTypeTok{fileEncoding =} \StringTok{"UTF-8"}\NormalTok{)}
\end{Highlighting}
\end{Shaded}

\begin{verbatim}
## Warning in write.csv(Data, "Data_Limpia.csv", sep = ",", dec = ".", fileEncoding
## = "UTF-8"): attempt to set 'sep' ignored
\end{verbatim}

\begin{verbatim}
## Warning in write.csv(Data, "Data_Limpia.csv", sep = ",", dec = ".", fileEncoding
## = "UTF-8"): attempt to set 'dec' ignored
\end{verbatim}

\end{document}
